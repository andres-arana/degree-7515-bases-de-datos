\documentclass[a4paper,11pt]{article}

%%%%%%%%%%%%%%%%%%%%%%%%%%%%%%%%%%%%%%%%%%%%%%%%%%%%%%%%%%%%%%%%%%%%%%%%
% Paquetes utilizados
%%%%%%%%%%%%%%%%%%%%%%%%%%%%%%%%%%%%%%%%%%%%%%%%%%%%%%%%%%%%%%%%%%%%%%%%

% Gráficos complejos
\usepackage{graphicx}
\usepackage{caption}
\usepackage{subcaption}
\usepackage{placeins}

% Soporte para el lenguaje español
\usepackage{textcomp}
\usepackage[utf8]{inputenc}
\usepackage[T1]{fontenc}
\DeclareUnicodeCharacter{B0}{\textdegree}
\usepackage[spanish]{babel}

% Código fuente embebido
\usepackage{listings}
\usepackage{courier}

% PDFs embebidos para el apéndice
\usepackage{pdfpages}

% Matemáticos
\usepackage{amssymb,amsmath}

% Tablas complejas
\usepackage{multirow}

% Formato de párrafo
\setlength{\parskip}{1ex plus 0.5ex minus 0.2ex}

% Subrayado de palabras
\usepackage[normalem]{ulem}

% Formato de listados de código
\lstset{%
  basicstyle=\footnotesize\ttfamily,
  numberstyle=\tiny,
  numbersep=5pt,
  tabsize=2,
  extendedchars=true,
  breaklines=true,
  stringstyle=\color{white}\ttfamily,
  showspaces=false,
  showtabs=false,
  xleftmargin=17pt,
  framexleftmargin=17pt,
  framexrightmargin=5pt,
  framexbottommargin=4pt,
  showstringspaces=false,
  language=SQL
}
\usepackage{caption}
\DeclareCaptionFont{white}{\color{white}}
\DeclareCaptionFormat{listing}{\colorbox[cmyk]{0.43, 0.35, 0.35,0.01}{\parbox{\textwidth}{\hspace{15pt}#1#2#3}}}
\captionsetup[lstlisting]{format=listing,labelfont=white,textfont=white, singlelinecheck=false, margin=0pt, font={bf,footnotesize}}

%%%%%%%%%%%%%%%%%%%%%%%%%%%%%%%%%%%%%%%%%%%%%%%%%%%%%%%%%%%%%%%%%%%%%%%%
% Título
%%%%%%%%%%%%%%%%%%%%%%%%%%%%%%%%%%%%%%%%%%%%%%%%%%%%%%%%%%%%%%%%%%%%%%%%

% Título principal del documento.
\title{\textbf{Trabajo Práctico: Hiposoft}}

% Información sobre los autores.
\author{%
  Andrés Gastón Arana,     \textit{P. 86.203}                      \\
  Diego Martín Costa,      \textit{P. 78.189}                      \\
  Javier Daniel Zaniratto, \textit{P. 90.886}                      \\
  Sergio Matías Piano,     \textit{P. 85.191}                      \\
  \\
  \normalsize{1er. Cuatrimestre de 2013}                           \\
  \normalsize{75.15 - Bases de datos}                              \\
  \normalsize{Facultad de Ingeniería, Universidad de Buenos Aires}
}
\date{}

%%%%%%%%%%%%%%%%%%%%%%%%%%%%%%%%%%%%%%%%%%%%%%%%%%%%%%%%%%%%%%%%%%%%%%%%
% Documento
%%%%%%%%%%%%%%%%%%%%%%%%%%%%%%%%%%%%%%%%%%%%%%%%%%%%%%%%%%%%%%%%%%%%%%%%

\begin{document}

% ----------------------------------------------------------------------
% Top matter
% ----------------------------------------------------------------------
\thispagestyle{empty}
\maketitle

\begin{abstract}

  Este informe sumariza el desarrollo del trabajo práctico de la materia Bases
  de Datos (75.15) dictada en el primer cuatrimestre de 2013 en la Facultad de
  Ingeniería de la Universidad de Buenos Aires. El mismo consiste en el
  modelado de datos de un software de administración de eventos hípicos en los
  hipódromos de la provincia de Buenos Aires, cuyos requisitos fueron extraídos
  de un caso de estudio real.

\end{abstract}

\clearpage

% ----------------------------------------------------------------------
% Tabla de contenidos
% ----------------------------------------------------------------------
\tableofcontents
\clearpage


% ----------------------------------------------------------------------
% Desarrollo
% ----------------------------------------------------------------------
\part{Desarrollo}

\section{Metodología y desarrollo}

Primeramente se analizó el enunciado entregado desarrollando un diagrama de
entidad-interrelación inicial plasmando el resultado del análisis.
Posteriormente se refinó el mismo al realizar una investigación de las
características del dominio del problema a resolver; particularmente, se
investigó en publicaciones hípicas de diversa procedencia acerca de las
diferentes relaciones entre jockeys, cuidadores y entrenadores con los studs
que a los que pertenecen los equinos, de manera de poder representar fielmente
en el diagrama las características de dichas relaciones.

Una vez completado y validado el diagrama de entidad interrelación, se
confeccionó el diccionario de datos, cuya elaboración fue trivial considerando
la existencia de dicho diagrama.

Posteriormente, se trabajó en el modelo relacional correspondiente al problema
relevado, mapeando cuidadosamente cada una de las entidades e interrelaciones a
las tablas. Se realizó en este caso también sucesivas aproximaciones al
diagrama final que se presenta en la sección correspondiente de este informe,
analizando las diversas estrategias de mapeo disponibles en cada caso y
seleccionando la más adecuada.

Finalmente, con el modelo relacional completo, se confeccionó un script SQL que
crea las tablas y restricciones correspondientes a dicho modelo utilizando el
subconjunto de instrucciones DDL del lenguaje.

\section{Modelo de entidad-interrelación} \label{sec:der}

\subsection{Hipótesis}

En la confección del diagrama se tomaron como verdaderas las siguientes
hipótesis:

\begin{enumerate}

  \item Los cuidadores y entrenadores trabajan sobre varios equinos al mismo
    tiempo. Cada cuidador y entrenador puede estar trabajando sobre diferentes
    equinos en un momento dado, aunque cada equino posee únicamente un cuidador
    y un entrenador.

  \item Diferentes jockeys pueden correr con diferentes equinos. Cada equino
    participa de una carrera con un jockey particular, pero eso no significa
    que en posteriores carreras deba seguir participando con el mismo jockey.
    Cada jockey puede participar en diferentes carreras con diferentes equinos.

  \item Entrenadores, cuidadores y jockeys trabajan para un stud particular, no
    pudiendo realizar sus tareas correspondientes en equinos de otros studs.

  \item Cada encuentro programado se realiza en su totalidad a lo largo de un
    único día.

\end{enumerate}

Todas estas hipótesis fueron confirmadas consultando a expertos y publicaciones
sobre el dominio del problema. Particularmente, en lo relacionado a la relación
entre los entrenadores, cuidadores y jockeys con los studs, se corroboró que en
el ámbito hípico estos trabajan para una y sólo una caballeriza. Siendo el
enunciado específico en definir stud como tanto el lugar donde reposan los
caballos como la caballeriza, esto indica que entonces el personal está
relacionado con los studs de la forma indicada.

\subsection{Diagrama de entidad-interrelación}

En la figura~\ref{fig:der} se incluye el diagrama de entidad-interrelación
final desarrollado para representar el dominio modelado cuyo relevamiento se
detalla en el enunciado.

\begin{figure}[h!t]
  \centering
  \includegraphics[width=1.5\textwidth, angle=90]{build/images/der.png}
  \caption{Diagrama de entidad-interrelación} \label{fig:der}
\end{figure}

\FloatBarrier

\subsection{Restricciones adicionales}

Las siguientes restricciones adicionales se detallan a continuación por
pertenecer al modelo de entidad-interrelación. No están representadas en el
diagrama dado que su inclusión agregaba complejidad adicional sobre en la
figura que consideramos innecesaria.

\begin{enumerate}

  \item La fecha del encuentro es única entre todos los encuentros.

  \item Cada caballo puede participar únicamente en una carrera por encuentro.
    Los jockeys pueden participar en cuantas carreras se quiera, en cada una de
    ellas con diferentes equinos.

\end{enumerate}

\section{Diccionario de datos}

\subsection{Hipódromo}

Representa una sociedad fundada para contribuir al desarrollo de la crianza de
equinos, que dispone de pistas utilizadas en encuentros programados en los
cuales se disputan las carreras de caballos.

\begin{itemize}

  \item \textbf{\uline{Nombre}} Nombre único descriptivo del hipódromo, como
    puede ser "Hipódromo de San Isidro".

  \item \textbf{Dirección} Dirección completa de la ubicación unívoca en donde
    se encuentran las pistas del hipódromo.

\end{itemize}

\subsection{Encuentro}

Representa un conjunto de carreras equinas que se disputaron o se disputarán de
acuerdo a la planificación anual en un único hipódromo en una fecha
determinada.

\begin{itemize}

  \item \textbf{\uline{Número}} Número correlativo con el que designa al
    encuentro.

  \item \textbf{Fecha} Fecha en la que se desarrollan las carreras del
    encuentro.

  \item \textbf{\dashuline{HipodromoNombre}} Nombre del hipódromo en el que se
    desarrollan las carreras del encuentro.

\end{itemize}

\subsection{Pista}

Representa la pista de los hipódromos en donde se realizarán carreras equinas.

\begin{itemize}

  \item \textbf{\uline{Nombre}} Nombre descriptivo de la pista con el que se
    designa la misma.

  \item \textbf{\uline{\dashuline{HipodromoNombre}}} Nombre del hipódromo del
    cual la pista forma parte.

  \item \textbf{Tipo} Indica el materia del piso de la pista, como césped y
    arena.

  \item \textbf{NumeroAndariveles} Cantidad de andariveles disponibles en la
    pista.

\end{itemize}

\subsection{Carrera}

Representa una competenecia en el que diferentes equinos y jockeys participan
intentando recorrer una distancia dada en una pista en el menor tiempo posible.

\begin{itemize}

  \item \textbf{\uline{Numero}} Número correlativo que identifica la carrera en
    el encuentro.

  \item \textbf{\uline{\dashuline{EncuentroNumero}}} Número del encuentro en el que está
    programada la carrera.

  \item \textbf{Tipo} Clasificación de la carrera de acuerdo a las reglas bajo
    las cuales se determina que equinos y jockeys pueden participar, como ser
    clásicas, handicap, perdedores y de grados.

  \item \textbf{EstadoPista} Indica el estado de la pista al momento de
    disputarse la carrera.

  \item \textbf{EstadoTiempo} Indica las condiciones climatológicas al momento
    de disputarse la carrera.

  \item \textbf{SexoEquinos} Sexo de los equinos que pueden participar en la
    carrera.

  \item \textbf{Distancia} Longitud pactada a recorrer para dar por completada
    la carrera, en metros.

  \item \textbf{Hora} Hora en la que está programada a comenzar la carrera.

  \item \textbf{\dashuline{PistaNombre}} Nombre de la pista en la que se disputa la
    carrera.

  \item \textbf{\dashuline{PistaHipodromoNombre}} Nombre del hipódromo en el cual se
    encuentra la pista en donde se disputa la carrera.

\end{itemize}

\subsection{Equino}

Representa a los equinos que podrán participar de las carreras a disputarse en
los encuentros, representando a un stud.

\begin{itemize}

  \item \textbf{\uline{Nombre}} Nombre identificatorio del equino.
    
  \item \textbf{Tipo} Indica la categoría del equino, la cual es utilizada
	para saber en que tipo de carreras puede participar (puede ser perdedor, 
	ganador de primera, de segunda, etc). 
  
  \item \textbf{Pelaje} Indica el color de pelaje que posee el equino (rojo, 
	negro, etc.).
  
  \item \textbf{Sexo} Indica el sexo del equino.
  
  \item \textbf{Handicap} Representa el handicap del equino que permite comparar
	sus ventajas o desventajas respecto a otros equinos.
   
  \item \textbf{\dashuline{StudNombre}} Identificador del stud a cual pertenece el equino.
  
  \item \textbf{\dashuline{PadreNombre}} Identificador del equino padre.
  
  \item \textbf{\dashuline{MadreNombre}} Identificador del equino madre.
  
  \item \textbf{\dashuline{CuidadorPersonaDNI}} Identificador de la persona que se encarga 
	del cuidado del equino (cumpliendo el rol de cuidador).
  
  \item \textbf{\dashuline{EntrenadorPersonaDNI}} Indentificador de la persona encargada 
	del entrenamiento del equino.
  
\end{itemize}




\subsection{Participación}

Representa la participación de un jockey y un equino en una carrera de un encuentro.
Se utiliza para planificar la participación, y, una vez concluida la carrera,  
guarda el tiempo que le tomó llegar a la meta, el andarivel en el que corrió,
el Peso y la Talla del jockey al momento de la carrera, el Handicap del equino
al momento de la carrera y si fue descalificado o no de la misma. 

\begin{itemize}

  \item \textbf{\uline{\dashuline{JockeyPersonaDNI}}} Identificador de la persona que montó
	al equino (jockey).
  
  \item \textbf{\uline{\dashuline{EquinoNombre}}} Identificador del equino participante.
  
  \item \textbf{\uline{\dashuline{CarreraNúmero}}} Identificador de la carrera.
  
  \item \textbf{\uline{\dashuline{CarreraEncuentroNúmero}}} Identificador del encuentro
        en el cual se había programado la carrera.
  
  \item \textbf{\dashuline{StudNombre}} Identificador del stud al cual pertenece el equino.
  
  \item \textbf{Tiempo} Indica el tiempo que tardó el equino en recorrer la distancia
	pactada para la carrera.
  
  \item \textbf{Andarivel} Indica el andarivel de la pista en el cual
	corrió el equino.
  
  \item \textbf{PesoJockey} Indica el peso del jockey en el momento de la carrera.
  
  \item \textbf{TallaJockey} Indica la talla del jockey cuando ocurrió la carrera.
  
  \item \textbf{HandicapEquino} Handicap del equino al momento de la carrera.
  
  \item \textbf{Descalificado} Indica si el equino fue descalificado o no
	de la carrera
  
\end{itemize}


\subsection{Cuidador}

Representa a las personas encargadas del cuidado de los equinos.

\begin{itemize}

        \item \textbf{\uline{\dashuline{PersonaDNI}}} DNI del cuidador.
	
\end{itemize}

\subsection{Entrenador}

Representa las personas que se encargan de entrenar a los equinos.

\begin{itemize}

        \item \textbf{\uline{\dashuline{PersonaDNI}}} DNI del entrenador.
	
\end{itemize}

\subsection{Persona}

Representa a una persona física.

\begin{itemize}

	\item \textbf{\uline{DNI}} DNI de la persona.
	
	\item \textbf{Nombres} Nombres de la persona.
	
	\item \textbf{Apellidos} Apellidos de la persona.
	
	\item \textbf{StudNombre} Identificador del stud para el cual trabaja la persona.
	
\end{itemize}

\subsection{Jockey}

Representa a los encargados de montar los equinos en las distintas carreras 
de los encuentros disputados en los hipódromos, representando a un stud.

\begin{itemize}

        \item \textbf{\uline{\dashuline{PersonaDNI}}} DNI del jockey.
	
	\item \textbf{Categoría} Categoría del Jockey (dependiendo de la cantidad
	de carreras ganadas puede ser aprendiz o profesional).
	
	\item \textbf{Peso} Peso actual del jockey.
	
	\item \textbf{Talla} Talla actual del jockey.

\end{itemize}

\subsection{Stud}

Representa las entidades dueñas de uno o varios equinos que correrán en los distintos
 encuentros, en representación del mismo.

\begin{itemize}

	\item \textbf{\uline{Nombre}} Identificador del stud.
	
	\item \textbf{ColorGorra} Color de la gorra que usarán los jockeys que 
	representen al stud en una carrera.
	
	\item \textbf{ColorChaqueta} Color de la chaqueta que usarán los jockeys 
	que representen al stud en una carrera.
	
\end{itemize}


\section{Resumen del diccionario}

\begin{itemize}

  \item \emph{Hipodromo}(\uline{Nombre}, Direccion)

  \item \emph{Encuentro}(\uline{Numero}, Fecha, \dashuline{HipodromoNombre})

  \item \emph{Pista}(\uline{Nombre}, \uline{\dashuline{HipodromoNombre}}, 
    Tipo, NumeroAndariveles)

  \item \emph{Carrera}(\uline{Numero}, \uline{\dashuline{EncuentroNumero}}, 
    Tipo, EstadoPista, EstadoTiempo, SexoEquinos, Distancia, \dashuline{PistaNombre}, 
    \dashuline{PistaHipodromoNombre}, Hora)

  \item \emph{Stud}(\uline{Nombre}, ColorChaqueta, ColorGorra)

  \item \emph{Persona}(\uline{DNI}, Nombres, Apellidos, \dashuline{StudNombre})

  \item \emph{Cuidador}(\uline{\dashuline{PersonaDNI}})

  \item \emph{Entrenador}(\uline{\dashuline{PersonaDNI}})

  \item \emph{Jockey}(\uline{\dashuline{PersonaDNI}}, Categoria, Peso, Talla)

  \item \emph{Equino}(\uline{Nombre}, Tipo, Pelaje, Sexo, Handicap, 
    \dashuline{StudNombre}, \dashuline{PadreNombre}, \dashuline{MadreNombre}, 
    \dashuline{CuidadorPersonaDNI}, \dashuline{EntrenadorPersonaDNI})

  \item \emph{Participacion}(\uline{\dashuline{JokeyPersonaDNI}}, 
    \uline{\dashuline{EquinoNombre}}, \uline{\dashuline{CarreraNumero}}, 
    \uline{\dashuline{CarreraEncuentroNumero}}, Tiempo, Andarivel, PesoJockey, 
    TallaJockey, HandicapEquino, \dashuline{StudNombre}, Descalificado)


\end{itemize}

\section{Modelo relacional}

\subsection{Diagrama relacional}

En la figura~\ref{fig:relacional} se incluye un diagrama representativo del
modelo relacional desarrollado para el modelo de entidad-interrelación
detallado en la sección~\ref{sec:der}.

\begin{figure}[h!t]
  \centering
  \includegraphics[width=1.65\textwidth, angle=90]{build/images/rel.png}
  \caption{Modelo relacional} \label{fig:relacional}
\end{figure}

\FloatBarrier

\subsection{Scripts de creación}

A continuación se incluyen los scripts de creación de las tablas del modelo
relacional para ser ejecutado en un sistema de bases de datos que interprete
SQL, particularmente las instrucciones DDL de dicho lenguaje. No se incluyen
las instrucciones, normalmente específicas al motor propiamente dicho, de
creación de usuarios, roles, schemas y bases de datos correspondientes.

\lstinputlisting{sql/schema.sql}

\clearpage

\part{Apéndice}
\appendix

\section{Enunciado original}\label{sec:enunciado}
\includepdf[pages={-}, frame=true, pagecommand={}, noautoscale=true, scale=0.7]{docs/enunciado.pdf}

\end{document}

